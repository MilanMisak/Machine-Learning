\documentclass[12pt]{article}
\usepackage{graphicx}
\usepackage[margin=0.8in]{geometry}
\begin{document}
{\bf Names:} Jack Bracewell, Milan Misak, Craig Ellis \\
{\bf Usernames:} jb2910, mm5510, ce710 \\
{\bf Group Number: 28}  \\ \\

\section*{Assignment 5: T-test}

\subsection*{Evaluation results}

\subsubsection*{Experimental results using clean and noisy data}
\begin{table}
\centering
\begin{tabular}{l | r r r}
Emotion   & Trees   & Networks & CBR     \\
\hline
Anger     & 28.4644 & 24.4726  & 24.0602 \\
Disgust   & 33.7349 & 60.0536  & 53.5484 \\
Fear      & 16.2393 & 46.3950  & 42.6036 \\
Happiness & 73.4793 & 77.4775  & 72.4211 \\
Sadness   & 33.5484 & 39.4366  & 34.4371 \\
Surprise  & 53.0612 & 75.4808  & 77.0563 \\
\end{tabular}
\caption{F1 measures}
\end{table}

\subsubsection*{T-test results using clean and noisy data}
\begin{table}
\centering
\begin{tabular}{l l | r r}
Emotion   &          & Clean Data & Noisy Data \\
\hline
Anger     & Tree/Net & 0          & 1          \\
          & Tree/CBR & 1          & 1          \\
          & Net/CBR  & 0          & 0          \\
Disgust   & Tree/Net & 0          & 0          \\
          & Tree/CBR & 0          & 0          \\
          & Net/CBR  & 0          & 0          \\
Fear      & Tree/Net & 1          & 1          \\
          & Tree/CBR & 1          & 0          \\
          & Net/CBR  & 0          & 0          \\
Happiness & Tree/Net & 0          & 1          \\
          & Tree/CBR & 0          & 0          \\
          & Net/CBR  & 0          & 1          \\
Sadness   & Tree/Net & 1          & 1          \\
          & Tree/CBR & 0          & 0          \\
          & Net/CBR  & 0          & 1          \\
Surprise  & Tree/Net & 1          & 0          \\
          & Tree/CBR & 1          & 0          \\
          & Net/CBR  & 0          & 0          \\
\end{tabular}
\caption{F1 measures}
\end{table}

\subsubsection*{T-test results using clean data}

\subsubsection*{T-test results using noisy data}


\subsubsection*{Which algorithm performed better overall in terms of values of $F_1$ measure (part I)? Which algorithm performed better when comparison was performed using the t-test (part II and part III)? Can we claim that this algorithm is a better learning algorithm than the others in general? Why? Why not?}

lorem ipsum.

\subsubsection*{How did you adjust the significance level in order to take into account the fact that you perform a multiple comparison test?}

We adjusted our chosen significance level (5 \%) by dividing it by 3 where 3 is the number of comparisons needed for this multiple comparison test.

\subsubsection*{Which type of t-test did you use and why?}

We used the paired T-test as the samples we were testing we not independent because they were coming from the exact same data.

\subsubsection*{Why do you think t-test was performed on the classification error and not the F1 measure?}

Something about classification error having higher varience maybe.

\subsubsection*{What is the trade-off between the number of folds you use and the number of examples per fold? In other words, what is going to happen if you use more folds, so you will have fewer examples per fold, or if you use fewer folds, so you will have more examples per fold?}

TODO

\subsubsection*{Suppose that we want to add some new emotions to the existing dataset. Which of the examined algorithms are more suitable for incorporating the new classes in terms of engineering effort? Which algorithms need to undergo radical changes in order to include new classes?}

Surely CBR is the easiest? Neural networks will need new optimal shit so that might be hardest?


\end{document}
